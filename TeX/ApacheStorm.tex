\documentclass{article}
\title{Apache Storm Notes}
\date{2015 Sept 10}
\date{2016 Jan 7}
\author{Milind Vaidya}

\begin{document}
    \maketitle
    \pagenumbering{gobble}
    \newpage
    \pagenumbering{arabic}
    \section{Rationale}
    \paragraph{}
    Realtime data processing at a massive(?) scale is what Apache Storm stands
    for in data processing ecosystem. It can be considered as "Hadoop for
    realtime". Before Storm, typically this was done manually with the help
    of workers and queues. The workers will pull out the message out of the
    queue and send it across or populate into DB. This approach was tedious
    and required lot of infrastructure building and was error prone. Storm
    provides a platform which does the heavy lifting where all these things
    are taken care of with the guaranteed message delivery, so that the
    focus can solely remain with the business logic. \section{Concepts}
    \subsection{Topology} \paragraph{} This is actual job at hand. It can be
    considered a capsule /  umbrella of business logic that is desired. This
    runs forever unless is killed forcibly.It is a graph of spouts and bolts
    connected by stream grouping. It is a graph of computation. Each node in
    a topology contains processing logic and links between nodes indicate
    how data should be passed around between the nodes.

    \subsection{Stream}
    \paragraph{}
    This is a channel of tuples. It is an unbound sequence of tuples that is
    generated and processed in parallel in a distributed fashion. A stream is
    defined by the schema that names the fields in the tuple which later take
    the form of name-value pairs. This is how actual message or data passing
    happens between the entities of a Topology such as bolts and spouts. In case
    of multi stream environment each stream can be given an id. But single
    stream topologies are also common, hence by default the name of the stream
    is set to "default"

    \subsection{Spout}
    \paragraph{}
    This is an interface of the Topology to the external world. This interacts
    with the external system/entity and acts as source of streams to the
    Topology. There are 2 types of spouts. The reliable ones keep record of each
    stream they send in to the topology and can reply if it fails or times out
    guaranteeing the reliable processing of the data (ack and fail methods). The
    unreliable ones on the other hand, forget about the tuple immediately once
    it leaves the spout.
    \paragraph{}
    \textbf{Declare a Stream:} declare method of the \textsf{OutputFieldsDeclarer}
    \paragraph{}
    \textbf{Emit the stream:} using emit method of \textsf{SpoutOutputCollector} 
    \paragraph{}
    \textbf{Produce new tuple:} \textsf{nextTuple} method of spout. DOES NOT BLOCK as all spout 
    methods are called in a single thread.

    \paragraph{}
     The different spout classes provide a convenient way to interface with
     corresponding producers such as Kestrel queue, Kafka etc.

    \subsection{Bolt}
    \paragraph{}
    This is heart of the topology as it contains actual processing. It is normal
    computational unit like a simple class which can perform any task. It
    subscribes to the stream of other component such as other bolt or spout. As
    it receives the stream tuple, it transforms it or appropriate action is
    taken on that tuple. They can also emit more than one stream which can later
    be consumed by other bolts.A bolt consumes any number of input streams, does
    some processing, and possibly emits new streams. Complex stream
    transformations, like computing a stream of trending topics from a stream of
    tweets, require multiple steps and thus multiple bolts. Bolts can do
    anything from run functions, filter tuples, do streaming aggregations, do
    streaming joins, talk to databases, and more. 
    \paragraph{}
    \textbf{Declare a stream :} \textsf{declareStream} method of 
    \textsf{OutputFieldsDeclarer} 
    \paragraph{}
    \textbf{Emit the stream :} emit method of \textsf{OutputCollector}
    The main method in a bolt is "execute" which can be anticipated.It has to "ack"
    every tuple by calling method having same name of \textsf{OutputCollector}that way storm
    knows that the tuple has been processed and then eventually original spout tuple
    can be acked. \textsf{OutputCollector} is not thread safe, so all the emits, acks and fails
    are supposed to happen in the same thread

    \subsection{Stream Grouping}
    \paragraph{}
    As we mentioned earlier that each of the bolts received tuples as it
    subscribes to a particular stream. The grouping defines how these tuples are
    given to the tasks in the bolt. In other words, a stream grouping tells a
    topology how to send tuples between two components.


    \subsection{Task}
    \paragraph{}
    Each spout or bolt is considered a task but there can be multiple instance
    of a tasks across a cluster. Each task corresponds to one thread of
    execution, and stream groupings define how to send tuples from one set of
    tasks to another set of tasks. e.g. SaveDataBolt may be connecting to DB and
    persisting the data. There will be multiple threads executing this task
    across the cluster such that each of them are  trying to connect to DB for
    pushing the data. (Still not clear if this is possible on the same machine
    leading to concurrency issues ? In other words can there be multiple threads
    of the same task in a worker ?)

    \subsection{Worker}
    \paragraph{}
    Topologies execute across one or more worker processes. Each worker process
    is a physical JVM and executes a subset of all the tasks for the topology.
    For example, if the combined parallelism of the topology is 300 and 50
    workers are allocated, then each worker will execute 6 tasks (as threads
    within the worker). Storm tries to spread the tasks evenly across all the
    workers \section{Storm Properties} Reliability : (Combine with the
    guaranteed processing)

    \section{Components of a Storm Cluster}
    \paragraph{}
    As mentioned briefly above, a topology does not stop, it has to be killed
    manually. A cluster facilitates the deployment of topology. It has 2 types
    of nodes, master nodes and worker nodes. 
    \paragraph{}
    \textbf{Master node Daemon : Nimbus :}
    distributes code along the cluster assigning tasks, monitoring failures etc.
    \paragraph{}
    \textbf{Worker node Daemon : Supervisor :}Listens for the task assigned, starts -
    stops workers as per the task assigned to it by Nimbus. 
    \paragraph{}
    Each worker executes a subset of a topology; a topology consists of many workers running across
    number of machines in the cluster. All coordination between Nimbus and the
    Supervisors is done through a Zookeeper cluster. Additionally, the Nimbus
    daemon and Supervisor daemons are fail-fast and stateless; all state is kept
    in Zookeeper or on local disk. This means you can kill -9 Nimbus or the
    Supervisors and they'll start back up like nothing happened. This design
    leads to Storm clusters being incredibly stable.

    \section{Defining Bolts in other languages}
    \paragraph{}
       Bolts can be defined in any language. Bolts written in another language
       are executed as subprocesses, and Storm communicates with those
       subprocesses with JSON messages over stdin/stdout. The communication
       protocol just requires an ~100 line adapter library, and Storm ships with
       adapter libraries for Ruby, Python, and Fancy 

    \section{Parallelism in Storm Topology}
    \paragraph{}
    The hierarchy of the working elements in the storm is as follows
    \begin{enumerate}
        \item \textbf{ Worker Process }
        \paragraph{}
        This maps directly to a spout or  bolt in the topology. These will be
        listed in the Storm UI as per the name assigned to each individual
        entity.
        \begin{enumerate}
            \item \textbf{ Executor }
            \paragraph{}
            This depends on the degree of parallelism mensioned at the time of
            configuration of the bolt. In our context it will be the value
            assigned to the parameter 'numThreads' in the config file for a
            particular bolt. This value controls max limit of parallelism. A
            given Worker process will have maximum 'numThreads' threads exeuting
            in parallel on various storm machines available in the cluster. This
            is indicated by 'Executors' column in the storm UI.
            \begin{enumerate}
                \item \textbf{ Task }
                \paragraph{}
                 A task performs the actual data processing — each spout or bolt
                 that you implement in your code executes as many tasks across
                 the cluster. The number of tasks for a component is always the
                 same throughout the lifetime of a topology, but the number of
                 executors (threads) for a component can change over time. This
                 means that the following condition holds true:\emph{threads}\leq \emph{tasks}
                 By default, the number of tasks is set to be the same
                 as the number of executors, i.e. Storm will run one task per
                 thread
            \end{enumerate}
        \end{enumerate}
    \end{enumerate}

\end{document}
